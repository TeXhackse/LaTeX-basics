\RequirePackage{scrletter}
\PassOptionsToPackage{table}{xcolor}
\RequirePackage{luacode}
\RequirePackage{xparse}
\newcounter{invoiceitem}

\RequirePackage{xltabular}
\RequirePackage{booktabs}
\RequirePackage{ragged2e}
\RequirePackage[round-mode=places,output-decimal-marker={,},round-integer-to-decimal]{siunitx}

\directlua{require("peitex-rechnung.lua")}

\NewDocumentCommand{\InitVAT}{om}{
	\luaexec{
	invoice.items[\luastringO{\romannumeral#2}]={}
	}
	\IfNoValueTF {#1}
		{
			\luaexec{invoice.vat[\luastringO{\romannumeral#2}]=#2/100}
		}{
			\luaexec{invoice.vat[\luastringO{\romannumeral#2}]=#1}
		}
}

%initialisierung auf 5,7,16 oder 19 % Mwst
\InitVAT{16}
\InitVAT{19}
\InitVAT{5}
\InitVAT{7}

\newcommand*{\SetDefaultVAT}[1]{\def\defaultVAT{#1}}

\newcommand*{\AddInvoiceItem}[4][\defaultVAT]{%
	\luaexec{invoice.additem(\luastringO{\romannumeral#1}, #2,
	"\luaescapestring{\unexpanded{#3}}",
#4 ) }
}


\newcolumntype{P}{S[round-precision=2,table-parse-only,table-number-alignment=right]<{\PrintTableCurrency}}

\newcommand*{\PrintInvoiceTabular}{
	\begin{xltabular}{\linewidth}{@{}rS[round-precision=1,table-format=2.1]>{\RaggedRight}XPP@{}}
	\toprule[\lightrulewidth]
	\noalign{\global\let\PrintTableCurrency\relax}%
	\small\emph{Pos.}&\small\emph{Std.}&\small\emph{Beschreibung}&\small\emph{Einzelpreis}&\small\emph{Gesamtpreis}\\\midrule[\heavyrulewidth]
	\noalign{\global\let\PrintTableCurrency\TableCurrency}%
	\endhead
	\bottomrule[\lightrulewidth]\multicolumn{5}{@{}p{\textwidth}@{}}{\strut\hspace*{\fill}\footnotesize Fortsetzung auf der nächsten Seite}\endfoot
	\bottomrule\endlastfoot
	\directlua{ invoice.printitems()}
	\midrule[\heavyrulewidth]
	\directlua{invoice.printsumvat()}
\end{xltabular}
}

%Ausgabe der einzelnen Rechnungspositionen
\newcommand*{\PrintInvoiceItem}[5][]{%
	\stepcounter{invoiceitem}%
	\theinvoiceitem%Positionsnummer
	&#2% Anzahl
	&#3\space(\printVAT{#1} MwSt.)% Beschreibung mit Angabe der MwSt, in Klammern
	&#4%Einzelpreis
	&#5%Gesamtpreis
	\tabularnewline
}

\newcommand*{\PrintInvoiceSum}[2]{
	\PrintSum{\csname invoicesum#1name\endcsname}{#2}
}

\newcommand*{\PrintVatSum}[3][]{
	\PrintSum{\invoicesumvatname[#1]{#2}}{#3}
}

\newcommand*{\invoicesumnettoname}{Summe (Netto)}
\newcommand*{\invoicesumbruttoname}{Summe (Brutto)}
\newcommand*{\invoicesumseparator}{:\space}
\newcommand*{\invoicesumvatname}[2][1]{MwSt. \printVAT{#2}\space	(\num[round-precision=2]{#1}\TableCurrency)}
\newcommand*{\TableCurrency}{\,€}
\newcommand*{\printVAT}[1]{\num[round-precision=0,round-integer-to-decimal=false]{#1}\,\%}

\renewcommand*{\theinvoiceitem}{\ifnum\value{invoiceitem}<10\relax0\fi\arabic{invoiceitem}}

\newcommand*{\PrintPositionenVAT}[5]{%
	\stepcounter{invoiceitem}%
	\theinvoiceitem&#1&#2\space(\printVAT{#3})&#4&#5\tabularnewline
}
	
\newcommand*{\PrintSum}[2]{
	&&\multicolumn{1}{r}{#1\invoicesumseparator}&\multicolumn{1}{l}{}&#2\tabularnewline
}

\endinput