% Kurzes kommentiertes scrletter-Beispiel
% © 2019 Marei Peischl, marei@texhackse.de
% 
\documentclass[
	ngerman,%Basissprache als globale Option
	]{scrartcl}

%Vorteil davon scrletter als Paket zu laden:
% Anlagen können direkt im Dokument mit bearbeitet werden. Alles was in Standard-\LaTeX existiert, gibt es hier auch.
\usepackage{scrletter}

%Unabhängigkeit gegenüber Compilern
\usepackage{iftex}
\ifPDFTeX
	\usepackage[utf8]{inputenc}%Falls TeX-Distro älter als Frühjahr 2018
	\usepackage{tgbonum}%Schriftart TeX Gyre Bonum
	\usepackage[T1]{fontenc}
\else
	\usepackage{fontspec}
	\setmainfont{TeX Gyre Bonum}%Schriftart TeX Gyre Bonum
\fi
%Alternative Schriftarten findet man unter https://tug.org/FontCatalogue/


\usepackage{babel}%Spracheinstellung, zusammen mit der Sprachoption bei documentclass

%Adressdaten aus hinterlegten Daten laden, hier lokale Datei adressdaten.lco.
%So muss man die nicht jedesmal neu machen.
%Varianten (privat/geschäftlich) und Installation im lokalen texmf tree möglich
\LoadLetterOption{adressdaten}

\begin{document}
	


	
\begin{letter}{Empfänger\\Empfängeradresse\\PLZ ort}
	
\setkomavar{subject}{Betreff}
%Bei geschäftlichen Dokumenten vielleicht auch
%\setkomavar{title}{Rechnung}

%Zusätzliche Einträge für die Geschäftszeile:
%date = \today, wenn nicht anders gesezt
%yourref, yourmail, myref, customer, invoice, place
%Alle nach Schema:
\setkomavar{customer}{1234567}
%Bezeichner können analog geändert werden, z.B.
%\setkomavar*{customer}{Mandantennummer}
	
\opening{Sehr geehrte Damen und Herren,}%Ohne Opening wird der Briefkopf nicht erzeugt => ggf. leer lassen.

Hier landet dann der eigentliche Text.

\closing{Mit freundlichen Grüßen}

%\ps PS: Hab was vergessen

%\encl{Anlage 1\\ Anlage 2}

%\cc{Ggf. Angaben zum Verteiler}

	
\end{letter}
\end{document}