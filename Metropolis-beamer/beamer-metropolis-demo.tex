\documentclass[
	aspectratio=169, %Breitbildformat
	]{beamer}


%Kompatibilität für alle Compiler
\usepackage{iftex}
\ifPDFTeX
\usepackage[utf8]{inputenc}%falls Distribution < TL2018
\usepackage[T1]{fontenc}
\fi

\usepackage[ngerman]{babel}% Sprachangabe -- fals english, ngerman entsprechend ersetzen


\usetheme[
	block=fill, %Blöcke sichtbar schalten
	]{metropolis}

\begin{document}
	
\titlegraphic{
%	So werden Bilder eingebunden
	\includegraphics[width=1.5cm]{example-image}
	% Wenn ihr eigene und keine Beispielbilder nehmt, dann sollte das Bild entweder im Gleichen ordner wie die .tex-Datei liegen oder in einem Unterordnet.
%	Falls Unterordner "Bilder":
%	\includegraphics[width=1.5cm]{Bilder/dein-eigenes-bild}
}
\title{Demo-Datei beamer}
\subtitle{Mit Verwendung des metropolis theme}
\author{Marei}
\institute{pei\TeX}

%Titelseite 
\frame{\titlepage}

\section{Abschnitt}
\begin{frame}{Folientitel}
Folie mit Titel
\end{frame}

\begin{frame}{Aufzählung}
\begin{itemize}
	\item eins
	\item zwei -- Ab hier wird schrittweise Aufgeblättert
	\pause %bewirkt, dass der folgende Inhalt ers ein Overlay später eingeblendet wird.
	\begin{itemize}
		\item nach der ersten pause
		\pause
		\item und noch eine	
	\end{itemize}
\end{itemize}
\end{frame}

\begin{frame}{Nummerierte Liste}
\begin{enumerate}
	\item eins
	\item zwei
\end{enumerate}
\end{frame}

\begin{frame}{großes Bild}
\centering%zentriert den inhalt der ganzen seite
%Bild wird geladen, skalierung über Breite ODER Höhe möglich. Hier in Abhängigkeit von der Zeilenbreite
\includegraphics[width=.9\linewidth]{example-image-16x9}
\end{frame}

\section{Platzierung}

\begin{frame}{Spalten}

%Umgebung für mehere Spalten
\begin{columns}
	%Spalte 1 mit 50% der Zeilenbreite
	\begin{column}{.5\linewidth}%Breite anteilig an zeilenbreite
	\includegraphics[width=\linewidth]{example-image}
	\end{column}

	%Spalte 2 mit 30%
	\begin{column}{.3\linewidth}
	%Man kann elemente auch ineinander Schachteln.
	\begin{itemize}
		\item eins
		\item zwei
	\end{itemize}
	\end{column}
\end{columns}

\end{frame}

\begin{frame}{Boxen}
\begin{block}{Blocktitel}
Inhalt
\end{block}

\begin{block}{Blocktitel}
	Inhalt
\end{block}

\end{frame}

\end{document}
